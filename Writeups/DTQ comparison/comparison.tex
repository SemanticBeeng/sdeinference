\documentclass[a4paper,11pt]{article}

\usepackage[left=2.5cm,right=2.5cm,top=3cm,bottom=3cm,pdftex]{geometry}
\usepackage{amssymb, amsmath, url, natbib, float, subcaption, listings,mathtools}
\usepackage[utf8]{inputenc}
\usepackage[T1]{fontenc}
\usepackage[pdftex]{graphicx}
\pdfcompresslevel=9 
\DeclareGraphicsExtensions{.png, .pdf, .jpg}
\usepackage[pdftex, colorlinks, linkcolor=blue, urlcolor=blue, citecolor=blue, pagecolor=blue, breaklinks=true]{hyperref}

\begin{document}
\pagestyle{empty}
\title{Comparison}
\begin{center}
\Large\textbf{CRAN packages comparison} \\[11pt]
\normalsize
\end{center}

\section{DiffusionRgqd}
Uses the cumulant truncation procedure developed by Varughese (2013), whereby the transition density can be approximated over arbitrarily large transition horizons for a suitably general class of non-linear diffusion models.

Generalized quadratic diffusions (GQD) are the specific class of SDEs with quadratic drift and diffusion terms:

\begin{align*}
d X_t & = \mu(X_t, t)dt + \sigma(X_t, t)dW_t, \: \text{where} \\
\mu(X_t, t) & = G_0(t) + G_1(t) X_t + G_2(t) X_t^2, \: \text{and} \\
\sigma (X_t, t) & = Q_0(t) + Q_1(t) X_t + Q_2(t) X_t^2
\end{align*}

For purposes of inference the drift and diffusion terms - and consequently the transitional density - are assumed to be dependent on a vector of parameters, $\theta$. For example, an Ornstein-Uhlenbeck model with SDE:

\begin{equation}
d X_t = \theta_1 (\theta_2 - X_t) + \sqrt{\theta_3^2} dW_t
\end{equation}

\begin{lstlisting}[language=R]
G0=function(t){theta[1]*theta[2]}
G1=function(t){-theta[1]}
Q0=function(t){theta[3]*theta[3]}
\end{lstlisting}

For a constant drift, diffusion SDE, with given initial condition $X_s$:
\begin{equation}
dX_t = \mu dt + \sigma dW_t
\end{equation}
The distribution at time $t$ of the process $X_t$ is $\mathcal{N}(X_t, X_s + \mu(t-s), \sigma^2(t-s))$

\begin{lstlisting}
Xs <- 0                 # Initial state
Xt <- seq(-3/2,3/2,1/50)# Possible future states
s  <- 0                 # Starting time
t  <- 1                 # Final time
mu    <- 0.5            # Drift parameter
sigma <- 0.25           # Diffusion coefficient

library(DiffusionRgqd) 
# Remove any existing coefficients:
GQD.remove()

# Define the model coefficients:
G0 <- function(t){mu}
Q0 <- function(t){sigma^2}

# Calculate the transitional density:
BM <- GQD.density(Xs,Xt,s,t)
\end{lstlisting}


\section{pomp: statistical inference for partially-observed Markov processes}
\section{Robfilter}
\section{Sim.DiffProc Package - FitSDE}
\section{HPloglik}
\section{abctools}

\end{document}

























